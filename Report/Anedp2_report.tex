\documentclass[12pt, a4paper]{article}
\usepackage[british]{babel}
\usepackage[utf8]{inputenc}
\usepackage{amsmath}
\usepackage{amssymb}
\usepackage{booktabs}
\usepackage{pgfplots}
\usepackage[en-US]{datetime2}
\usepackage[hidelinks]{hyperref}
\pgfplotsset{/pgf/number format/use comma,compat=newest}

\title{\textbf{Discontinuous Galerkin FE approximation of elliptic problems on polyhedral grids}}
\author{Andrea Vescovini\\[1cm]{\small Supervisor: Prof. P. Antonietti}}
\DTMlangsetup{showdayofmonth=false}
\date{\today}
%%%%%%%%%%%%%%%%%%%%%%%%%%%%%%%%%%%%%%%%%%%%%%%%%%%%%%%%%%%%%%%%%%%%%%%%%%

\begin{document}
\maketitle
\newpage
\begin{abstract}
	The main goal of this project is to implement a Discontinuous Galerkin (DG) method for solving a three-dimensional Poisson problem with Dirichlet boundary conditions, employing a general polyhedral mesh.\\
	DG methods have shown to be very flexible and have been successfully applied to hyperbolic, elliptic and parabolic problem arising from many different fields of application.
	Moreover one of the main advantages with respect to the continuous framework is the possibility of handling meshes with hanging nodes and made of general-shaped elements without too many difficulties.\\
	In section~\ref{sec:DG} we recover the main results about standard DG methods, then in section~\ref{sec:poly} we develop the theory in order to handle polyhedral grids. Subsequently in section~\ref{sec:imp} we explain our main choices for the algorithm implementation and in section~\ref{sec:res} we present the numerical results we obtained in some prototypal cases.
\end{abstract}
\phantomsection
\tableofcontents
\newpage

%%%%%%%%%%%%%%%%%%%%%%%%%%%%%%%%%%%%%%%%%%%%%%%%%%%%%%%%%%%%%%%%%%%%%%%%%

%\section*{Introduction}
%Introduzione sì o introduzione no?

\section{DG finite elments methods}\label{sec:DG}
In this section we follow mainly chapter 2 of \cite{riviere} and chapter 11 of \cite{quart}.
\subsection{Model problem}
Let's consider a Poisson problem with Dirichlet boundary conditions
\begin{align} \label{eq:poisson}
	-\Delta u = f & \mbox{ in } \Omega\\
			u = g & \mbox{ on } \partial \Omega
\end{align}
where $\Omega \subset \mathbf{R}^3$ is a bounded polyhedral domain with a Lipschitz boundary $\partial \Omega$, the source $f$ belongs to $L^2(\Omega)$ and Dirichlet datum $g$ belongs to $H^{1/2}(\partial \Omega)$.
The usual weak formulation is:\\
Find $u \in H^1(\Omega)$ such that $u = \tilde{u} + R_g$, with $\tilde{u} \in H^1_0(\Omega)$ such that:
\begin{equation} \label{eq:wform}
	\int_\Omega \nabla u \cdot \nabla v
	= \int_\Omega fv - \int_\Omega \nabla R_g \cdot \nabla v, \;\; \forall v \in H^1_0,
\end{equation}
and $R_g \in H^1(\Omega)$ is a lifting of $g$, i.e. $R_g|_{\partial \Omega} = g$.\\
DG methods make use of a variational formulation different from the usual one so we have to introduce new spaces in which we will look for the solution.
%%%%%%%%%%%%%%%%%%%%%%%%%%%%%%%%%%%%%%%%%%%%%%%%%%%%%%%%%%%%%%%%%%%%%%%%%%%%%%%%
\subsection{Broken Sobolev spaces}
Let $\mathcal{T}$ be a subdivision of $\Omega$ into disjoint elements $\kappa$ such that $ \bar{\Omega} = \bigcup\limits_{\kappa \in \mathcal{T}} \bar{\kappa}$, let $h_\kappa$ be the diameter of the element $\kappa$ i.e. $h_\kappa = \min\limits_{x, y \in \kappa} |x-y|$ and let $\rho_\kappa$ be the maximum diameter of a ball inscribed in $\kappa$. For the moment we assume $\mathcal{T}$ to be regular, i.e. that $\exists C > 0$ such that:
\begin{equation*}
	\frac{h_\kappa}{\rho_\kappa} < C, \; \; \forall \kappa \in \mathcal{T}.
\end{equation*}
We define for every real number $s$ the broken Sobolev space:
\begin{equation*}
	H^s(\mathcal{T}) = \{ v \in L^2(\Omega) : v|_\kappa \in H^s(\kappa), \forall \kappa \in \mathcal{T} \},
\end{equation*}
with the norm:
\begin{equation*}
	|||v|||_{H^s(\mathcal{T})} = \bigg( \sum_{\kappa \in \mathcal{T}} ||v||_{H^s(\kappa)}^2 \bigg)^{1/2}.
\end{equation*}
We denote by $\Gamma_h$ the set of interior faces and by $\Gamma_D$ the set of faces that are on the boundary $\partial \Omega$. For every face $e \in \Gamma_h$ there are two elements $\kappa^+$ and $\kappa^-$ that share it and they both have their outward normal $\mathbf{n}^+$ and $\mathbf{n}^-$.\\
Since every function $v$ of $H^s(\mathcal{T})$ is well defined along any side of every $\kappa$, if $e \in \Gamma_h$ there are two different traces of $v$ along $e$ and we denote them by $v^+$ and $v^-$; it will be useful to introduce jumps and average of these traces so we can define:
\begin{align*}
	\{v\} = \frac{1}{2} (v^+ + v^-) ,
	& \;\; [v] = v^+ \mathbf{n}^+ + v^- {n}^-,\\
	\{\!\!\{ \mathbf{v} \}\!\!\} = \frac{1}{2} (\mathbf{v}^+ +\mathbf{v}^-),
	& \;\; [\![ \mathbf{v} ]\!] = \mathbf{v}^+ \cdot \mathbf{n}^+ + \mathbf{v}^- \cdot \mathbf{n}^-,
\end{align*}
that can be extended to $e \in \Gamma_D$ through:
\begin{equation*}
	\{v\} = v, \;\; [v] = v \mathbf{n}^,
	\;\; \{\!\!\{ \mathbf{v} \}\!\!\} = \mathbf{v}, \;\; [\![ \mathbf{v} ]\!] = \mathbf{v} \cdot \mathbf{n}.
\end{equation*}
Notice that the above definitions are independent of which element is called "$^+$" and which "$^-$".
%%%%%%%%%%%%%%%%%%%%%%%%%%%%%%%%%%%%%%%%%%%%%%%%%%%%%%%%%%%%%%%%%%%%%%%%%%%%%%
\subsection{Variational formulation}
In what follows we assume that the weak solution $u$ of the Poisson problem~ \eqref{eq:wform} belongs to $H^s(\mathcal{T})$ with $s > 3/2$, so that the following calculations will be meaningful.\\
Integrating \eqref{eq:poisson} by parts we can obtain:
\begin{equation} \label{eq:green}
	\sum_{\kappa \in \mathcal{T}} \int_\kappa -\Delta u \; v
	= \sum_{\kappa \in \mathcal{T}} \bigg( \int_\kappa \nabla u \cdot \nabla v
	- \int_{\partial \kappa} v \nabla u \cdot \mathbf{n} \bigg), \;\; \forall v \in H^s(\mathcal{T}).
\end{equation}
With some manipulations, we can see that:
\begin{equation} \label{eq:jumps}
\begin{split}
	\sum_{\kappa \in \mathcal{T}} \int_{\partial \kappa} v \nabla u \cdot \mathbf{n} &= \sum_{e \in \Gamma_h \cup \Gamma_D} \int_e (v^+ \nabla u^+ \cdot \mathbf{n}^+ + v^- \nabla u^- \cdot \mathbf{n}^- )\\
	&= \sum_{e \in \Gamma_h \cup \Gamma_D} \int_e ([v] \cdot \{\!\!\{ \nabla u \}\!\!\} + [\![ \nabla u ]\!] \{v\} ), \;\; \forall v \in H^s(\mathcal{T}).
\end{split}
\end{equation}
Then inserting~\eqref{eq:jumps} into~\eqref{eq:green} we can obtain that the solution of the Poisson problem~\eqref{eq:pois} is $u \in H^s(\mathcal{T})$ such that:
\begin{equation} \label{eq:firstform}
	\sum_{\kappa \in \mathcal{T}} \int_\kappa \nabla u \cdot \nabla v -
	\sum_{e \in \Gamma_h \cup \Gamma_D} \int_e ([v] \cdot \{\!\!\{ \nabla u \}\!\!\} + [\![ \nabla u ]\!] \{v\} ) =
	\sum_{\kappa \in \mathcal{T}} \int_\kappa fv, \; \forall v \in H^s(\mathcal{T}).
\end{equation}
At this point we have to remember that if the exact solution $u \in H^s(\mathcal{T})$, then $[\![\nabla u]\!] = 0$ and $[u] = 0$ on every internal face $e \in \Gamma_h$, so in~\eqref{eq:firstform} the term~$[\![ \nabla u ]\!] \{v\}$ is null and we can add on the left hand side two tunable term that will give to the formulation more stability:
\begin{equation*}
	\epsilon \sum_{e \in \Gamma_h} \int_e [u] \cdot \{\!\!\{ \nabla v \}\!\!\},
\end{equation*}
\begin{equation*}
	\sum_{e \in \Gamma_h} \frac{\sigma_e}{|e|^\beta} \int_e [u][v],
\end{equation*}
where $\epsilon = \{-1, 0, 1\}$ is a parameter that will affect the symmetry of the formulation, $|e|$ is the 2D-measure of the face $e$, $\sigma_e$ and $\beta$ are two parameters that will be specified later and will be important for the well posedness of the problem.\\
Finally we impose the Dirichlet boundary condition in a weak way, as it is more natural for a DG method, so we add on the left:
\begin{equation*}
	\epsilon \sum_{e \in \Gamma_D} \int_e (u-g) \nabla v \cdot \mathbf{n}
	+ \sum_{e \in \Gamma_D} \frac{\sigma_e}{|e|^\beta} \int_e (u-g)v.
\end{equation*}
We have obtained the general DG variational formulation:\\
Find $u \in H^s(\mathcal{T}), s>3/2$, such that:
\begin{multline} \label{eq:dgvarform}
	\sum_{\kappa \in \mathcal{T}} \int_\kappa \nabla u \cdot \nabla v
	-\sum_{e \in \Gamma_h \cup \Gamma_D} \bigg( \int_e [v] \cdot \{\!\!\{ \nabla u \}\!\!\}
	-\epsilon \int_e [u] \cdot \{\!\!\{ \nabla v \}\!\!\}
	+ \frac{\sigma_e}{|e|^\beta} \int_e [u][v] \bigg)\\
	= \sum_{\kappa \in \mathcal{T}} \int_\kappa fv
	+ \sum_{e \in \Gamma_D} \bigg( \epsilon \int_e g \nabla v \cdot \mathbf{n}
	+ \frac{\sigma_e}{|e|^\beta} \int_e gv \bigg), \;\; \forall v \in H^s(\mathcal{T}).
\end{multline}

\section{Polyhedral grids}\label{sec:poly}

\section{Implementation}\label{sec:imp}
	
\section{Numerical results}\label{sec:res}

\section{Conclusion}\label{sec:conc}

%%%%%%%%%%%%%%%%%%%%%%%%%%%%%%%%%%%%%%%%%%%%%%%%%%%%%%%%%%%%%%%%%%%%%%%%%%
\newpage
\begin{thebibliography}{9}
	\bibitem{review}
	Antonietti P. F., Cangiani  A., Collins J., Dong Z., Georgoulis E. H., Giani S., Houston P.: Review of discontinuous Galerkin finite element methods for partial differential equations on complicated domains. \emph{Building bridges: connections and challenges in modern approaches to numerical partial differential equations, Lecture Notes in Computational Science and Engineering}, volume 114, 279–307 (2016).
	
	\bibitem{mox}
	Antonietti, P.F., Facciola, C., Russo, A., Verani, M.: Discontinuous Galerkin approximation of flows in fractured porous media on polygonal and polyhedral meshes. \emph{MOX Report} 55/2016 (2016).
	
	\bibitem{multigrid}
	Antonietti P. F., Houston  P., Hu  X., Sarti  M., Verani M.: Multigrid algorithms for hp-version interior penalty discontinuous Galerkin methods on polygonal and	polyhedral meshes. \emph{Calcolo}, 1-30 (2017).
	
	\bibitem{hpmet}
	Cangiani A., Georgoulis E. H., Houston P.: Hp-Version discontinuous Galerkin methods on polygonal and polyhedral meshes. \emph{Mathematical Models and Methods in Applied Sciences}, 24(10), 2009-2041 (2014).
	
	\bibitem{hest}
	Hesthaven J. S., Warburton T.: \emph{Nodal Discontinuous Galerkin Methods: Algorithms, Analysis and Applications}. Springer, Berlin (2007).
	
	\bibitem{paper4}
	Giani S., Houston P.: Domain decomposition preconditioners for discontinuous Galerkin discretizations of compressible fluid flows. \emph{Numerical Mathematics: Theory, Methods and
	Applications}, 7(2), 123-128 (2014).
	
	\bibitem{quart}
	Quarteroni A.: \emph{Numerical Models for Differetial Problems}. MS\&A, Springer-Verlag Italia, Milan (2014).
	
	\bibitem{riviere}
	Rivière B.: \emph{Discontinuous Galerkin Methods for Solving Elliptic and Parabolic Equations: Theory and Implementation}, volume 35 of \emph{Frontiers in Applied Mathematics}. Society for Industrial and Applied Mathematics (SIAM), Philadelphia, PA (2008).
\end{thebibliography}

\end{document}