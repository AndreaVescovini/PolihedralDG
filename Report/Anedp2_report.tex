\documentclass[12pt, a4paper]{article}
\usepackage[british]{babel}
\usepackage[utf8]{inputenc}
\usepackage{amsmath}
\usepackage{amssymb}
\usepackage{amsthm}
\usepackage{booktabs}
\usepackage{pgfplots}
\usepackage[en-US]{datetime2}
\usepackage[hidelinks]{hyperref}
\pgfplotsset{/pgf/number format/use comma,compat=newest}

\theoremstyle{definition}
\newtheorem{ipotesi}{Assumption}

\theoremstyle{plain}
\newtheorem{lemma}{Lemma}

\theoremstyle{plain}
\newtheorem{teor}{Theorem}

\title{\textbf{Discontinuous Galerkin FE approximation of elliptic problems on polyhedral grids}}
\author{Andrea Vescovini\\[1cm]{\small Supervisor: Prof. P. Antonietti}}
\DTMlangsetup{showdayofmonth=false}
\date{\today}
%%%%%%%%%%%%%%%%%%%%%%%%%%%%%%%%%%%%%%%%%%%%%%%%%%%%%%%%%%%%%%%%%%%%%%%%%%

\begin{document}
\maketitle
\newpage
\begin{abstract}
	The main goal of this project is to implement a Discontinuous Galerkin (DG) method for solving a three-dimensional Poisson problem with Dirichlet boundary conditions, employing a general polyhedral mesh.\\
	DG methods have shown to be very flexible and have been successfully applied to hyperbolic, elliptic and parabolic problem arising from many different fields of application.
	Moreover one of the main advantages with respect to the continuous framework is the possibility of handling meshes with hanging nodes and made of general-shaped elements without too many difficulties.\\
	In section~\ref{sec:DG} we recover the main results about standard DG methods, then in section~\ref{sec:poly} we develop the theory in order to handle polyhedral grids. Subsequently in section~\ref{sec:imp} we explain our main choices for the algorithm implementation and in section~\ref{sec:res} we present the numerical results we obtained in some prototypal cases.
\end{abstract}
\phantomsection
\tableofcontents
\newpage

%%%%%%%%%%%%%%%%%%%%%%%%%%%%%%%%%%%%%%%%%%%%%%%%%%%%%%%%%%%%%%%%%%%%%%%%%

%\section*{Introduction}
%Introduzione sì o introduzione no?

\section{DG finite elments methods}\label{sec:DG}
In this section we follow mainly chapter 2 of \cite{riviere} and chapter 11 of \cite{quart}.
\subsection{Model problem}
Let's consider a Poisson problem with Dirichlet boundary conditions
\begin{align} \label{eq:poisson}
	-\Delta u = f & \mbox{ in } \Omega\\
			u = g & \mbox{ on } \partial \Omega
\end{align}
where $\Omega \subset \mathbb{R}^3$ is a bounded polyhedral domain with a Lipschitz boundary $\partial \Omega$, the source $f$ belongs to $L^2(\Omega)$ and Dirichlet datum $g$ belongs to $H^{1/2}(\partial \Omega)$.
The usual weak formulation is:\\
Find $u \in H^1(\Omega)$ such that $u = \tilde{u} + R_g$, with $\tilde{u} \in H^1_0(\Omega)$ such that:
\begin{equation} \label{eq:wform}
	\int_\Omega \nabla u \cdot \nabla v
	= \int_\Omega fv - \int_\Omega \nabla R_g \cdot \nabla v, \quad \forall v \in H^1_0,
\end{equation}
and $R_g \in H^1(\Omega)$ is a lifting of $g$, i.e. $R_g|_{\partial \Omega} = g$.\\
DG methods make use of a variational formulation different from the usual one so we have to introduce new spaces in which we will look for the solution.
%%%%%%%%%%%%%%%%%%%%%%%%%%%%%%%%%%%%%%%%%%%%%%%%%%%%%%%%%%%%%%%%%%%%%%%%%%%%%%%%
\subsection{Broken Sobolev spaces}
Let $\mathcal{T}$ be a subdivision of $\Omega$ into disjoint open tetrahedral elements $\kappa$ such that $\bar{\Omega} = \bigcup\limits_{\kappa \in \mathcal{T}} \bar{\kappa}$, let $h_\kappa$ be the diameter of the element $\kappa$ i.e. $h_\kappa = \min\limits_{x, y \in \kappa} |x-y|$ and let $\rho_\kappa$ be the maximum diameter of a ball inscribed in $\kappa$. For the moment we assume $\mathcal{T}$ to be regular, i.e. that $\exists C > 0$ such that:
\begin{equation*}
	\frac{h_\kappa}{\rho_\kappa} < C, \; \; \forall \kappa \in \mathcal{T}.
\end{equation*}
We define for every real number $s$ the broken Sobolev space:
\begin{equation*}
	H^s(\mathcal{T}) = \{ v \in L^2(\Omega) : v|_\kappa \in H^s(\kappa), \forall \kappa \in \mathcal{T} \},
\end{equation*}
with the norm:
\begin{equation*}
	|\!|\!|v|\!|\!|_{H^s(\mathcal{T})} = \bigg( \sum_{\kappa \in \mathcal{T}} |\!|v|\!|_{H^s(\kappa)}^2 \bigg)^{1/2}.
\end{equation*}
By extension $L^2(\mathcal{T})$ can be seen as $H^0(\mathcal{T})$. Moreover it holds that:
\begin{equation*}
	H^s(\Omega) \subset H^s(\mathcal{T}) \quad \text{ and } \quad H^{s+1}(\mathcal{T}) \subset H^s(\mathcal{T})
\end{equation*}
We denote by $\Gamma_h$ the set of interior faces and by $\Gamma_D$ the set of faces that are on the boundary $\partial \Omega$, we define $\Gamma = \Gamma_h \cup \Gamma_D$. For every face $e \in \Gamma_h$ there are two elements $\kappa^+$ and $\kappa^-$ that share it and they both have their outward normal $\mathbf{n}^+$ and $\mathbf{n}^-$.\\
Since every function $v$ of $H^s(\mathcal{T})$ is well defined along any side of every $\kappa$, if $e \in \Gamma_h$ there are two different traces of $v$ along $e$ and we denote them by $v^+$ and $v^-$; it will be useful to introduce jumps and average of these traces so we can define:
\begin{align*}
	\{v\} = \frac{1}{2} (v^+ + v^-) ,
	& \quad [v] = v^+ \mathbf{n}^+ + v^- {n}^-,\\
	\{\!\!\{ \mathbf{v} \}\!\!\} = \frac{1}{2} (\mathbf{v}^+ +\mathbf{v}^-),
	& \quad [\![ \mathbf{v} ]\!] = \mathbf{v}^+ \cdot \mathbf{n}^+ + \mathbf{v}^- \cdot \mathbf{n}^-,
\end{align*}
that can be extended to $e \in \Gamma_D$ through:
\begin{equation*}
	\{v\} = v, \quad [v] = v \mathbf{n}^,
	\quad \{\!\!\{ \mathbf{v} \}\!\!\} = \mathbf{v}, \quad [\![ \mathbf{v} ]\!] = \mathbf{v} \cdot \mathbf{n}.
\end{equation*}
Notice that the above definitions are independent of which element is called "$^+$" and which "$^-$".
%%%%%%%%%%%%%%%%%%%%%%%%%%%%%%%%%%%%%%%%%%%%%%%%%%%%%%%%%%%%%%%%%%%%%%%%%%%%%%
\subsection{Variational formulation}
In what follows we assume that the weak solution $u$ of the Poisson problem~ \eqref{eq:wform} belongs to $H^s(\mathcal{T})$ with $s > 3/2$, so that the following calculations will be meaningful.\\
Integrating \eqref{eq:poisson} by parts we can obtain:
\begin{equation} \label{eq:green}
	\sum_{\kappa \in \mathcal{T}} \int_\kappa -\Delta u \; v
	= \sum_{\kappa \in \mathcal{T}} \bigg( \int_\kappa \nabla u \cdot \nabla v
	- \int_{\partial \kappa} v \nabla u \cdot \mathbf{n} \bigg), \quad \forall v \in H^s(\mathcal{T}).
\end{equation}
With some manipulations, we can see that:
\begin{equation} \label{eq:jumps}
\begin{split}
	\sum_{\kappa \in \mathcal{T}} \int_{\partial \kappa} v \nabla u \cdot \mathbf{n} &= \sum_{e \in \Gamma} \int_e (v^+ \nabla u^+ \cdot \mathbf{n}^+ + v^- \nabla u^- \cdot \mathbf{n}^- )\\
	&= \sum_{e \in \Gamma} \int_e ([v] \cdot \{\!\!\{ \nabla u \}\!\!\} + [\![ \nabla u ]\!] \{v\} ), \quad \forall v \in H^s(\mathcal{T}).
\end{split}
\end{equation}
Then inserting~\eqref{eq:jumps} into~\eqref{eq:green} we can obtain that the solution of the Poisson problem~\eqref{eq:poisson} is $u \in H^s(\mathcal{T})$ such that:
\begin{equation} \label{eq:firstform}
	\sum_{\kappa \in \mathcal{T}} \int_\kappa \nabla u \cdot \nabla v -
	\sum_{e \in \Gamma} \int_e ([v] \cdot \{\!\!\{ \nabla u \}\!\!\} + [\![ \nabla u ]\!] \{v\} ) =
	\sum_{\kappa \in \mathcal{T}} \int_\kappa fv, \; \forall v \in H^s(\mathcal{T}).
\end{equation}
At this point we have to remember that if the exact solution $u \in H^s(\mathcal{T})$, then $[\![\nabla u]\!] = 0$ and $[u] = 0$ on every internal face $e \in \Gamma_h$, so in~\eqref{eq:firstform} the term~$[\![ \nabla u ]\!] \{v\}$ is null and we can add on the left hand side two tunable term that will give to the formulation more stability:
\begin{equation*}
	\epsilon \sum_{e \in \Gamma_h} \int_e [u] \cdot \{\!\!\{ \nabla v \}\!\!\},
\end{equation*}
\begin{equation*}
	\sum_{e \in \Gamma_h} \gamma_e \int_e [u][v], \quad \gamma_e = \frac{\sigma_e}{|e|^\beta}
\end{equation*}
where $\epsilon = \{-1, 0, 1\}$ is a parameter that will affect the symmetry of the formulation, $|e|$ is the $2$-dimensional measure of the face $e$, $\sigma_e$ and $\beta$ are two parameters that will be specified later and will be important for the well~posedness of the problem.\\
Finally we impose the Dirichlet boundary condition in a weak way, as it is more natural for a DG method, so we add on the left:
\begin{equation*}
	\epsilon \sum_{e \in \Gamma_D} \int_e (u-g) \nabla v \cdot \mathbf{n}
	+ \sum_{e \in \Gamma_D} \gamma_e \int_e (u-g)v.
\end{equation*}
We have obtained the general DG variational formulation:\\
Find $u \in H^s(\mathcal{T}), s>3/2$, such that:
\begin{multline*}
	\sum_{\kappa \in \mathcal{T}} \int_\kappa \nabla u \cdot \nabla v 
	-\sum_{e \in \Gamma} \bigg( \int_e [v] \cdot \{\!\!\{ \nabla u \}\!\!\}
	-\epsilon \int_e [u] \cdot \{\!\!\{ \nabla v \}\!\!\}
	- \gamma_e \int_e [u][v] \bigg)\\
	= \sum_{\kappa \in \mathcal{T}} \int_\kappa fv
	+ \sum_{e \in \Gamma_D} \bigg( \epsilon \int_e g \nabla v \cdot \mathbf{n}
	+ \gamma_e \int_e gv \bigg), \quad \forall v \in H^s(\mathcal{T}).
\end{multline*}
Viceversa it can be proved (\cite{riviere}) that if the solution $u$ of \eqref{eq:dgvarform} belongs to $H^1(\Omega) \cap H^s(\mathcal{T})$, then $u$ is the solution of the Poisson problem \eqref{eq:poisson}.\\
We can introduce the bilinear form $a_\epsilon:~H^s(\mathcal{T})~\times~H^s(\mathcal{T})~\rightarrow~\mathbb{R}$:
\begin{equation*}
a_\epsilon(u, v) = \sum_{\kappa \in \mathcal{T}} \int_\kappa \nabla u \cdot \nabla v
-\sum_{e \in \Gamma} \bigg( \int_e [v] \cdot \{\!\!\{ \nabla u \}\!\!\}
-\epsilon \int_e [u] \cdot \{\!\!\{ \nabla v \}\!\!\}
- \gamma_e \int_e [u][v] \bigg)
\end{equation*}
and the functional $F_\epsilon:~H^s(\mathcal{T})~\rightarrow~\mathbb{R}$:
\begin{equation*}
	F_\epsilon(v) = \sum_{\kappa \in \mathcal{T}} \int_\kappa fv
	+ \sum_{e \in \Gamma_D} \bigg( \epsilon \int_e g \nabla v \cdot \mathbf{n}
	+ \gamma_e \int_e gv \bigg),
\end{equation*}
so that we can rewrite the variational formulation in a compact fashion:\\
Find $u \in H^s(\mathcal{T}), s>3/2$, such that:
\begin{equation} \label{eq:dgvarform}
	a_\epsilon(u, v) = F_\epsilon(v), \quad \forall v \in H^s(\mathcal{T}).
\end{equation}
%%%%%%%%%%%%%%%%%%%%%%%%%%%%%%%%%%%%%%%%%%%%%%%%%%%%%%%%%%%%%%%%%%%%%%%%%%%
\subsection{Discrete formulation}
We introduce now the finite dimensional subspace of $H^s(\mathcal{T})$, for $s>3/2$:
\begin{equation} \label{eq:dgspace}
	\mathcal{D}_r(\mathcal{T}) = \{ v \in L^2(\Omega) : v|_\kappa \in \mathbb{P}_r(\kappa), \; \forall \kappa \in \mathcal{T}  \},
\end{equation}
where $\mathbb{P}_r(\kappa)$ denotes the space of polynomials of total degree less then or equal to $r$ over the element $\kappa$.\\
We endow it with the so called \textit{energy norm}:
\begin{equation*}
	|\!|v_h|\!|_{DG} = \bigg( |\!|\nabla v_h|\!|^2_{L^2(\mathcal{T})} + \sum_{e \in \Gamma} \gamma_e \int_e [v_h]^2 \bigg)^{1/2}.
\end{equation*}
It's easy to see that it is a norm if $\gamma_e > 0 \quad \forall e \in \Gamma$.\\
So the general DG finite element formulation is:\\
Find $u_h \in \mathcal{D}_r(\mathcal{T})$ such that:
\begin{equation} \label{eq:dgfemform}
	a_\epsilon(u_h, v_h) = F_\epsilon(v_h), \quad \forall v_h \in \mathcal{D}_r(\mathcal{T}).
\end{equation}
This kind of formulation is called \textit{Interior Penalty} (IP) and as briefly mentioned before the parameter $\epsilon$ can influence the symmetry:
\begin{itemize}
\item
If $\epsilon = -1$ the method is called \textit{Symmetric Interior Penalty Galerkin} (SIPG), indeed in the bilinear becomes symmetrical because the two terms involving jumps and averages appear with the same sign.
\item
If  $\epsilon = +1$ the method is instead called \textit{Non-symmetric Interior Penalty Galerkin} (NIPG), for an opposite reason.
\item
If $\epsilon = 0$ the method is called \textit{Incomplete Interior Penalty Galerkin} (IIPG).
\end{itemize}
%%%%%%%%%%%%%%%%%%%%%%%%%%%%%%%%%%%%%%%%%%%%%%%%%%%%%%%%%%%%%%%%%%%%%%%%%%%
\subsection{Well posedness}
In order to prove the well posedness of \eqref{eq:dgfemform} we have to show coercivity and continuity of the bilinear form and continuity of the functional with respect to $\mathcal{D}_r(\mathcal{T})$ with the norm $||\cdot||_{DG}$, then apply the Lax-Milgram lemma.\\
We will use the following inverse trace inequality for polynomials (\cite{riviere}, p. 23):
\begin{equation} \label{eq:trineq}
	|\!|v|\!|_{L^2(e)} \leq \frac{\bar{C}}{\sqrt{h_\kappa}} |\!|v|\!|_{L^2(\kappa)}, \quad \forall e \subset \partial \kappa, \; \forall v \in \mathbb{P}_r(\kappa)
\end{equation}
%%%%%%%%%%%%%%%%%%%%%%%%%%%%%%%%%%%%%%%%%%%%%%%%%%%%%%%%%%%%%%%%%%%%%%%%%%
\subsubsection{Coercivity}
We have to show that $\exists \alpha > 0 $ such that:
\begin{equation*}
a_\epsilon(u_h, u_h) = |\!|u_h|\!|^2_{DG} + (\epsilon - 1) \sum_{e \in \Gamma} \int_e [u_h] \cdot \{\!\!\{ \nabla u_h \}\!\!\} \geq \alpha |\!|u_h|\!|^2_{DG}.
\end{equation*}
If $\epsilon = 1$ we get immediately the coercivity with a coercivity constant $\alpha = 1$.
If $\epsilon = \{0,-1\}$ more care is needed: using Cauchy-Schwarz's inequality and Young's inequality we obtain $\forall \delta > 0$:
\begin{equation} \label{pas:young}
\bigg| \sum_{e \in \Gamma} \int_e [u_h] \cdot \{\!\!\{ \nabla u_h \}\!\!\} \bigg| \leq
\frac{\delta}{2} \sum_{e \in \Gamma} \bigg( \frac{1}{\gamma_e} \bigg)  \big|\!\big| \{\!\!\{ \nabla u_h \}\!\!\} \big|\!\big|^2_{L^2(e)}
+ \frac{1}{2\delta} \bigg|\!\bigg| \sqrt{\gamma_e} [u_h] \bigg|\!\bigg|^2_{L^2(\Gamma)}
\end{equation}
Then we observe that:
\begin{equation*}
\big|\!\big| \{\!\!\{ \nabla u_h \}\!\!\} \big|\!\big|^2_{L^2(e)} \leq
\begin{cases}
\big|\!\big| \nabla u_h \big|\!\big|^2_{L^2(e)}, & \text{if } e \in \Gamma_D\\
\frac{1}{4} \big( \big|\!\big| \nabla u_h|_{\kappa^+} \big|\!\big|^2_{L^2(e)} + \big|\!\big| \nabla u_h|_{\kappa^-} \big|\!\big|^2_{L^2(e)} \big), & \text{if } e \in \Gamma_h
\end{cases}
\end{equation*}
and use the inverse trace inequality \eqref{eq:trineq}:
\begin{equation*}
\big|\!\big| \nabla u_h \big|\!\big|^2_{L^2(e)} \leq \frac{\bar{C}^2}{h_\kappa} \big|\!\big| \nabla u_h \big|\!\big|^2_{L^2(\kappa)}.
\end{equation*}
Now we exploit the fact that trivially $|e| < h_\kappa^{d-1}, \; \forall e \subset \partial \kappa$, where $d$ is the dimension in which we are working, i.e. $3$:
\begin{multline} \label{pas:invineq}
\sum_{e \in \Gamma} \bigg( \frac{|e|^\beta}{\sigma_e} \bigg)  \big|\!\big| \{\!\!\{ \nabla u_h \}\!\!\} \big|\!\big|^2_{L^2(e)}
\leq \sum_{e \in \Gamma_h} \frac{\bar{C}^2}{4\sigma_e} \bigg( h_{\kappa^+}^{2\beta - 1} \big|\!\big| \nabla u_h \big|\!\big|^2_{L^2(\kappa^+)} + h_{\kappa^-}^{2\beta - 1} \big|\!\big| \nabla u_h \big|\!\big|^2_{L^2(\kappa^-)} \bigg)\\
+ \sum_{e \in \Gamma_D} \frac{\bar{C}^2}{\sigma_e} h_\kappa^{2\beta - 1} \big|\!\big| \nabla u_h \big|\!\big|^2_{L^2(\kappa)}
\leq \sum_{\kappa \in \mathcal{T}} \sum_{e \in \partial \kappa} \frac{\bar{C}^2}{\sigma_e} \big|\!\big| \nabla u_h \big|\!\big|^2_{L^2(\kappa)}
\leq \frac{n^*\bar{C}^2}{\sigma^*} \big|\!\big| \nabla u_h \big|\!\big|^2_{L^2(\mathcal{T})}.
\end{multline}
After having assumed that that $2\beta - 1 \geq 0$, supposed without loss of generality that $h = \min\limits_{\kappa \in \mathcal{T}} h_\kappa < 1$, defined $\sigma^* = \min\limits_{e \in \Gamma} \sigma_e$ and $n^*$ the number of faces for every element.\\
Putting together \eqref{pas:young} and \eqref{pas:invineq} obtaining:
\begin{equation*}
\begin{split}
a_\epsilon(u_h, u_h) &\geq \big(1 - \frac{n^*\delta\bar{C}^2}{2\sigma^*} (1-\epsilon)\big) \big|\!\big| \nabla u_h \big|\!\big|^2_{L^2(\mathcal{T})}
+ \big(1 - \frac{1}{2\delta} \big) \bigg|\!\bigg| \sqrt{\gamma_e} [u_h] \bigg|\!\bigg|^2_{L^2(\Gamma)}\\
&\geq \min\bigg\{1 - \frac{n^*\delta\bar{C}^2}{2\sigma^*} (1-\epsilon) , 1 - \frac{1-\epsilon}{2\delta}\bigg\} |\!|u_h|\!|^2_{DG}
\end{split}
\end{equation*}
Choosing for example $\delta = 1$ for $\epsilon = 0$ and  $\delta = 1/2$ for $\epsilon = -1 $ we need then $\sigma^* > \frac{n^*\bar{C}^2}{2}$ in order to get the coercivity constant $\alpha > 0$.
%%%%%%%%%%%%%%%%%%%%%%%%%%%%%%%%%%%%%%%%%%%%%%%%%%%%%%%%%%%%%%%%%%%%%%%%%%
\subsubsection{Continuity}
We have to show that $\exists M_1 > 0$ such that:
\begin{equation*}
	a_\epsilon(u_h, v_h) \leq M_1 |\!|u_h|\!|_{DG} |\!|v_h|\!|_{DG}, \quad \forall u_h, v_h \in \mathcal{D}_r(\mathcal{T})
\end{equation*}
and that $\exists M_2 > 0$ such that:
\begin{equation*}
	F_\epsilon(v_h) \leq M_2 |\!|v_h|\!|_{DG}, \quad \forall v_h \in \mathcal{D}_r(\mathcal{T})
\end{equation*}
The proof follows easily from standard arguments; indeed exploiting triangular inequality and Cauchy-Schwarz's inequality we can obtain very straight forward the required bounds for the first and last term of $a_\epsilon(u, v)$. For the second and third term we just have to use smartly Cauchy-Schwarz and then apply the trace inequality \eqref{eq:trineq} like for the coercivity.\\
We can proceed analogously for $F_\epsilon(v)$.
%%%%%%%%%%%%%%%%%%%%%%%%%%%%%%%%%%%%%%%%%%%%%%%%%%%%%%%%%%%%%%%%%%%%%%%%%%
\subsubsection{Existence and uniqueness}
Finally we have that always for NIPG and if $\beta \geq 1/2$ and $\sigma^*$ is greater enough for SIPG and IIPG, then Lax-Milgram lemma holds so the solution $u_h$ of \eqref{eq:dgfemform} exists and is unique, moreover:\\
\begin{equation*}
	|\!|u_h|\!|_{DG} \leq \frac{M_2}{\alpha},
\end{equation*}
where $M_2$ is the continuity constant of $F_\epsilon(v_h)$ and $\alpha$ is the coercivity constant of $a_\epsilon(u_h,v_h)$ and thy depend on $f$, $g$ and $\bar{C}$.
%%%%%%%%%%%%%%%%%%%%%%%%%%%%%%%%%%%%%%%%%%%%%%%%%%%%%%%%%%%%%%%%%%%%%%%%%%
\subsubsection{Mass conservation}
An interesting property of DG methods useful for transport problems is the local conservation of mass, indeed if we choose a test function that is $1$ only on one interior element $\kappa^+$ and $0$ elsewhere we get:
\begin{equation*}
	- \int_{\partial \kappa^+} \{\!\!\{ \nabla u_h \}\!\!\} \cdot \mathbf{n} + \sum_{e \in \partial \kappa^+} \gamma_e \int_e u_h|_{\kappa^+} - u_h|_{\kappa^-} = \int_{\kappa^+} f
\end{equation*}
that can be interpreted as a balance equation in which the first term is the outgoing flux and the second term in a mathematical mass that is zero if the penalty parameter $\sigma_e$ is zero.
%%%%%%%%%%%%%%%%%%%%%%%%%%%%%%%%%%%%%%%%%%%%%%%%%%%%%%%%%%%%%%%%%%%%%%%%%%%
\subsection{Error analysis}
We quote from \cite{riviere} error bounds both in the $DG$-norm and in the $L^2$-norm.
\begin{teor} \label{teo:errdg}
	Assume that the exact solution to the problem~\eqref{eq:wform} belongs to $H^k(\mathcal{T}), \; k>3/2$. Assume also $\beta = 1/2$ and that the penalty parameter $\sigma_e$ is large enough for the SIPG and IIPG methods. Then $\exists C>0$ independent of~$h$ such that:
	\begin{equation*}
		|\!| u -u_h |\!|_{DG} \leq h^{s-1} |\!|\!|u|\!|\!|_{H^k(\mathcal{T})}.
	\end{equation*}
	where $s = \min \{r+1, k\}$.
\end{teor}
\begin{teor}
	Assume that theorem~\ref{teo:errdg} holds. Then for the SIPG method $\exists C>0$ independent of $h$, such that:
	\begin{equation*}
		|\!| u-u_h |\!|_{L^2(\Omega)} \leq C h^{s} |\!|\!|u|\!|\!|_{H^k(\mathcal{T})},
	\end{equation*}
	while for both th NIPG and IIPG methods the following suboptimal estimate holds:
	\begin{equation*}
				|\!| u-u_h |\!|_{L^2(\Omega)} \leq C h^{s-1} |\!|\!|u|\!|\!|_{H^k(\mathcal{T})},
	\end{equation*}
	It has been observed numerically on uniform meshes that for NIPG and IIPG convergence rates are optimal if the polynomial degree is odd and suboptimal if it is even
\end{teor}
%%%%%%%%%%%%%%%%%%%%%%%%%%%%%%%%%%%%%%%%%%%%%%%%%%%%%%%%%%%%%%%%%%%%%%%%%%%
\section{Polyhedral grids}\label{sec:poly}
In this section we follow mainly \cite{multigrid} and \cite{hpmet}.
\subsection{Grid assumptions}
Suppose that now $\mathcal{T}$ is a partition of the computational domain $\Omega$ into disjoint open polyhedral elements $\kappa$ such that $\bar{\Omega} = \bigcup_{\kappa \in \mathcal{T}} \bar{\kappa}$. In order to allow the presence of hanging nodes/edges, we define the \textit{interfaces} of $\mathcal{T}$  as the intersection of the $(d-1)$-dimensional facets of neighbouring elements. Then, being us in $d=3$, we assume that each interface of an element $\kappa \in \mathcal{T}$ may be subdivided by a set of co-planar triangles. So with \textit{face} we refer to a $(d-1)$-dimensional simplex (i.e. a triangle in $d=3$), which forms part of an interface of an element $\kappa \in \mathcal{T}$; this means that each interfaces has a sub-triangulation into faces. As before with $\Gamma_h$ we denote the union of all open faces of the mesh interior to $\Omega$ and with $\Gamma_D$ the union of all open faces of the mesh contained in the boundary $\partial \Omega$; then $\Gamma = \Gamma_h \cup \Gamma_D$.\\
We need now some assumptions on the mesh $\mathcal{T}$:
\begin{ipotesi} \label{ipo:ipo1}
	Given $\kappa \in \mathcal{T}$, there exists a set of non-overlapping (not necessarily shape-regular) $3$-dimensional simplices $T_j \subset \kappa, \; j = 1,\dots, n_\kappa$, such that for any face $e \subset \partial \kappa$, $\bar{e} = \partial \bar{\kappa} \cap \partial \bar{T_j}$, for some $j$,
	\begin{equation*}
		\cup_{j = 1}^{n_\kappa} \bar{T_k} \subseteq \bar{\kappa},
	\end{equation*}
	and  $\exists C > 0$ such that the diameter $h_\kappa$ of $\kappa$ can be bounded by:
	\begin{equation*}
		h_\kappa \leq C \frac{3 |T_j|}{|e|}, \quad \forall j = 1,\dots,n_\kappa.
	\end{equation*}
\end{ipotesi}
Observe that we are not restricting neither the number of faces nor the measure of a face with respect to the measure of the element.
\begin{ipotesi} \label{ipo:ipo2}
	For any element $\kappa \in \mathcal{T}$ we assume that $\exists C > 0$ such that $h^3_\kappa~\geq~|\kappa|~\geq~Ch^3_\kappa$.
\end{ipotesi}
\begin{ipotesi} \label{ipo:ipo3}
	Every element $\kappa \in \mathcal{T}$ admits a sub-triangulation into at most $m_\kappa \in \mathbb{N}$ non-overlapping, shape-regular simplices $\mathit{s}_i, \; i = 1,\dots,m_\kappa$, such that:
	\begin{equation*}
		\bar{\kappa} = \bigcup\limits_{i = 1}^{m_\kappa} \bar{\mathit{s}_i}, \quad \text{ and } \exists C>0 \text{ such that }  \quad |\mathit{s}_i| \geq C |\kappa|, \; \forall i = 1,\dots,m_\kappa,
	\end{equation*}
	with $C$ independent of $\kappa$.
\end{ipotesi}
\begin{ipotesi} \label{ipo:ipo4}
	Let $\mathcal{T}^\# = \{ \mathcal{K} \}$ be a covering of $\Omega$ made of shaped-regular $3$-dimensional simplices $\mathcal{K}$. We assume that for any $\kappa~\in~\mathcal{T}, \; \exists\mathcal{K}\in\mathcal{T}^\# $ such that $\kappa\subset\mathcal{K}$, $\exists~C_d>0$ such that:
	\begin{equation*}
		diam(\mathcal{K})\leq~C_dh_\kappa, \quad \text{uniformly with respect to the mesh size}
	\end{equation*}
	and $\exists~C_c>0$ such that:
	\begin{equation*}
		\max\limits_{\kappa \in \mathcal{T}} card \big\{ \kappa' \in \mathcal{T} : \kappa' \cap \mathcal{K} \ne \emptyset, \; \mathcal{K} \in \mathcal{T}^\# \text{ such that } \kappa \subset \mathcal{K} \big\} \leq C_c.
	\end{equation*}
\end{ipotesi}
Notice that in this way the mesh regularity is assumed for the mesh covering $\mathcal{T}^\#$ and not for the computational mesh $\mathcal{T}$.
\begin{ipotesi} \label{ipo:ipo5}
	The mesh $\mathcal{T}$ is quasi uniform, i.e. $\exists C>0$ such that $h~\leq~C \min_{\kappa \in \mathcal{T}} h_\kappa$.
\end{ipotesi}
%%%%%%%%%%%%%%%%%%%%%%%%%%%%%%%%%%%%%%%%%%%%%%%%%%%%%%%%%%%%%%%%%%%%%%%%%%%
\subsection{Well posedness}
The discrete formulation is the same of \eqref{eq:dgfemform}, but we analyse only the SIPG case ($\epsilon~=~-1$) and we choose the penalty parameter $\gamma_e \in L^\infty(\mathcal{T})$ such that:
\begin{equation*}
	\gamma_e =
	\begin{cases}
		\sigma \max\limits_{\kappa \in \{\kappa^+, \kappa^-\}} \big\{ \frac{r^2}{h_\kappa}\big\},
		& \quad e \in \Gamma_h, \; e \subset \partial\kappa^+ \cap \partial\kappa^-,\\
		\sigma\frac{r^2}{h_\kappa},& \quad e \in \Gamma_D, \; e \subset \partial\kappa^+ \cap \partial\Omega,
	\end{cases}
\end{equation*}
where $\sigma$ is a positive constant independent of $r$, $|e|$ and $|\kappa|$ (we remind that $r$ is the degree of polynomials in the space $\mathcal{D}_r(\mathcal{T})$ \eqref{eq:dgspace}.\\
So it becomes:\\
Find $u_h \in \mathcal{D}_r(\mathcal{T})$ such that:
\begin{equation} \label{eq:dgfempolyform}
	a_{-1}(u_h, v_h) = F_{-1}(v_h), \quad \forall v_h \in \mathcal{D}_r(\mathcal{T}).
\end{equation}
We will refer to $a_{-1}(\cdot, \cdot)$ as simply $a\cdot, \cdot)$ and to $F_{-1}(\cdot)$ as $F(\cdot)$.\\
The proof of well posedness through continuity and coercivity can be found in \cite{hpmet} under slightly weaker assumptions on the mesh, but it follows the same path of the one with classical meshes discussed in section~\ref{sec:DG}, concluding again that we have to require $\sigma$ to be large enough. It exploits the following variation of the inverse trace inequality \eqref{eq:trineq}, whose proof can be found in \cite{multigrid}:
\begin{lemma}
	Assume that the mesh $\mathcal{T}$ satisfies Assumption \ref{ipo:ipo1}. Let $\kappa \in \mathcal{T}$ be a polyhedral element, then the following bound holds:
	\begin{equation*}
		|\!|v|\!|^2_{L^2(\partial\kappa)} \leq C_{inv} \frac{r^2}{h_\kappa} |\!|v|\!|^2_{L^2(\kappa)}, \quad \forall v \in \mathbb{P}_r(\kappa),
	\end{equation*}
	where $C_{inv}$ in a constant independent of $|\kappa|$, $r$ and $v$.
\end{lemma}
%%%%%%%%%%%%%%%%%%%%%%%%%%%%%%%%%%%%%%%%%%%%%%%%%%%%%%%%%%%%%%%%%%%%%%%%%%%
\subsection{Error analysis}
We will make use of an extension operator $\mathcal{E}:H^k(\Omega)\rightarrow H^k(\mathbb{R}^d), \; k\in\mathbb{N}_0$, continuos, such that $\mathcal{E}v|_\Omega=v$ (chapter 7, \cite{salsa}).\\
We now present the following approximation lemma from \cite{multigrid}:
\begin{lemma}  \label{lemma:interp}
	Given Assumptions \ref{ipo:ipo1} and \ref{ipo:ipo4}, let $v~\in~H^k(\mathcal{T}), \; k>3/2$ such that $\mathcal{E}v|_\mathcal{K}~\in~H^k(\mathcal{K})$, for each $\kappa~\in~\mathcal{T}$, where $\kappa~\subset~\mathcal{K}, \; \mathcal{K}~\in~\mathcal{T}^\#$. Then there exists a projection operator $\bar{\Pi}:~L^2(\Omega)~\rightarrow~\mathcal{D}_r(\mathcal{T})$ such that:
	\begin{equation}
		\big|\!\big| v - \bar{\Pi}v \big|\!\big|_{DG}
		\leq C_{interp} \frac{h^{s-1}}{r^{k-1-\mu/2}} |\!| v |\|_{H^k(\Omega)},
	\end{equation}
	where $s = \min \{r+1, k\}$, $C_{interp}$ depends on the shape-regularity constant $C_d$ of the covering $\mathcal{T}^\#$, but is independent of the discretization parameters, as well as the number of faces per element and the relative measure of the faces. Here $\mu = 0$ whenever a $r$-optimal interpolant can be constructed and $\mu = 1$ otherwise.
\end{lemma}
Thanks to this lemma we can state the following error bound both in the $L^2$-norm and $DG$-norm:
\begin{teor}
	Given Assumptions \ref{ipo:ipo1} and \ref{ipo:ipo4}, let $u_h \in \mathcal{D}_r(\mathcal{T})$ be the DG-solution of problem~\eqref{eq:dgfempolyform}. If the exact solution of \eqref{eq:wform} is such that $u\in H^k(\mathcal{T}), \; k>5/2$, such that $\mathcal{E}v|_\mathcal{K}~\in~H^k(\mathcal{K})$, for each $\kappa~\in~\mathcal{T}$, where $\kappa~\subset~\mathcal{K}, \; \mathcal{K}~\in~\mathcal{T}^\#$, then the following bounds hold:
	\begin{equation} \label{eq:dgbound}
		|\!|u-u_h|\!|_{DG} \leq G \frac{h^{s-1}}{r^{k-1-\mu /2}} |\!|u|\!|_H^k(\Omega),
	\end{equation}
	\begin{equation} \label{eq:l2bound}
		|\!|u-u_h|\!|_{L^2(\Omega)} \leq C_{L^2} \frac{h^s}{r^{k-\mu}} |\!|u|\!|_H^k(\Omega),
	\end{equation}
	where $s = \min \{r+1, k\}$ and the constants $G$, $C_{L^2}$ are independent of the discretization parameters. Here $\mu = 0$ whenever a $r$-optimal interpolant can be constructed and $\mu = 1$ otherwise.
\end{teor}
Notice that we are assuming only Assumptions \ref{ipo:ipo1} and \ref{ipo:ipo4}, that do not put restrictions on the number or size of faces.
\begin{proof}
	The error bound \eqref{eq:dgbound} follows from the general theorem proved in \cite{hpmet}, we will instead prove the bound~\eqref{eq:l2bound}.\\
	Let $w \in H^2(\Omega)$ be the solution of the problem:
	\begin{equation*}
		a(v, w) = \int_{\Omega} (u-u_h)v, \quad \forall v \in  H^2(\Omega))
	\end{equation*}
	Using standard elliptic regularity theorems we know that $\exists C_{reg}>0$ such that:
	\begin{equation} \label{eq:reg}
		|\!| w |\!|_{H^2(\Omega)} \leq C_{reg} |\!| u - u_h |\!|_{L^2(\Omega)}.
	\end{equation}
	The exploiting Galerkin orthogonality:
	\begin{equation} \label{eq:go}
	\begin{split}
		|\!| u - u_h |\!|^2_{L^2(\Omega)} &= a(u-u_h, w)\\
		&= a(u-u_h, w-w_h) \leq M_1 |\!| u - u_h |\!|_{DG} |\!| w - w_h |\!|_{DG},
	\end{split}
	\end{equation}

	For any function $w_h \in \mathcal{D}_r(\mathcal{T})$; so we can select $w_h = \bar{\Pi}w$ and employ lemma~\ref{lemma:interp} followed by~\eqref{eq:reg}:
	\begin{equation*}
		|\!|w-\bar{\Pi}w|\!|_{DG} \leq C_{interp} \frac{h}{r^{1-\mu/2}} |\!| w |\!|_{H^2(\Omega)} \leq C_{interp}C_{reg} \frac{h}{r^{1-\mu/2}} |\!| u-u_h |\!|_{L^2(\Omega)}.
	\end{equation*}
	Inserting this into \eqref{eq:go} gives the desired result \eqref{eq:l2bound}.
\end{proof}	
Finally we quote from \cite{multigrid} the following result about the maximum eigenvalue of the bilinear form $a(\cdot, \cdot)$:
\begin{teor}
	Under Assumptions \ref{ipo:ipo1}, \ref{ipo:ipo2}, \ref{ipo:ipo3} and \ref{ipo:ipo5}, for any function $u_h \in \mathcal{D}_r(\mathcal{T}) $ we have that:
	\begin{equation*}
		a(u_h, u_h) \leq C_{eig} \frac{r^4}{h^2} |\!| u_h |\!|^2_{L^2(\mathcal{T})}.
	\end{equation*}
\end{teor}
%%%%%%%%%%%%%%%%%%%%%%%%%%%%%%%%%%%%%%%%%%%%%%%%%%%%%%%%%%%%%%%%%%%%%%%%%%
\section{Implementation}\label{sec:imp}
	
\section{Numerical results}\label{sec:res}

\section{Conclusion}\label{sec:conc}

%%%%%%%%%%%%%%%%%%%%%%%%%%%%%%%%%%%%%%%%%%%%%%%%%%%%%%%%%%%%%%%%%%%%%%%%%%
\newpage
\begin{thebibliography}{99}
	\bibitem{review}
	Antonietti P. F., Cangiani  A., Collins J., Dong Z., Georgoulis E. H., Giani S., Houston P.: Review of discontinuous Galerkin finite element methods for partial differential equations on complicated domains. \emph{Building bridges: connections and challenges in modern approaches to numerical partial differential equations, Lecture Notes in Computational Science and Engineering}, volume 114, 279–307 (2016).
	
	\bibitem{mox}
	Antonietti, P. F., Facciolà, C., Russo, A., Verani, M.: Discontinuous Galerkin approximation of flows in fractured porous media on polygonal and polyhedral meshes. \emph{MOX Report} 55/2016 (2016).
	
	\bibitem{multigrid}
	Antonietti P. F., Houston  P., Hu  X., Sarti  M., Verani M.: Multigrid algorithms for hp-version interior penalty discontinuous Galerkin methods on polygonal and	polyhedral meshes. \emph{Calcolo}, 1-30 (2017).
	
	\bibitem{hpmet}
	Cangiani A., Georgoulis E. H., Houston P.: Hp-Version discontinuous Galerkin methods on polygonal and polyhedral meshes. \emph{Mathematical Models and Methods in Applied Sciences}, 24(10), 2009-2041 (2014).
	
	\bibitem{dunavant}
	Dunavant D.: High degree efficient symmetrical Gaussian quadrature rules for the triangle. \emph{International Journal for Numerical Methods in Engineering}, 21, 1129-1148 (1985).
	
	\bibitem{hest}
	Hesthaven J. S., Warburton T.: \emph{Nodal Discontinuous Galerkin Methods: Algorithms, Analysis and Applications}. Springer, Berlin (2007).
	
	\bibitem{paper4}
	Giani S., Houston P.: Domain decomposition preconditioners for discontinuous Galerkin discretizations of compressible fluid flows. \emph{Numerical Mathematics: Theory, Methods and Applications}, 7(2), 123-128 (2014).
	
	\bibitem{quart}
	Quarteroni A.: \emph{Numerical Models for Differetial Problems}. MS\&A, Springer-Verlag Italia, Milan (2014).
	
	\bibitem{salsa}
	Salsa S.: \emph{Partial Differential Equations in Action: From Modelling to Theory}. Springer (2016).
	
	\bibitem{riviere}
	Rivière B.: \emph{Discontinuous Galerkin Methods for Solving Elliptic and Parabolic Equations: Theory and Implementation}, volume 35 of \emph{Frontiers in Applied Mathematics}. Society for Industrial and Applied Mathematics (SIAM), Philadelphia, PA (2008).
\end{thebibliography}

\end{document}